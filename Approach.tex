
  How I want to get to the Solution and why I choosed it also how it works.
  Also there should be some theory about state estimation and the kalmanfilter used
  in this
  
  \section{Tipps/Notes}
  - Don't forget point of origin / reference system
  - Don't forget the pitch angle
  Maybe do rocket model here
 
  
  \subsection{Barometer}
  -Use density as a statevector \\
  -Use exponential atmosphere method. \\
  -Increasing the R measurement noise matrix when rocket is ascending access the rising uncertainties
  
  
  \section{Possibilities}
  Which ones are there and what are their positives and negatives, in short how do they work
  I will use kalman filter with time depending system and sensor noise, but i have yet to define how i do this in peticullar.
  I should also make some pictures and such things.
  
  There are many possibilities to do sensor fusion. first of all an all new algorithm could be developed which accesses the 
  stated problems directly. While this solution would be preferebal regarding the efficency, the 
  time ad knowledge needed for this task would exceed the resoursces given in this thesis by far.
  Also as stated in chapter \ref{ch:Introduction} a lot of theoretical as well as practical pre work is
  already done and therefore should be used. 
  
  \subsection{Kalmanfilter}
  The discrete Kalmanfilter was first introduced by Kalman in the year 1960.
  Its structure provides the optimal estimation of the standard deviation estimation error as long as the noise is Gaussian.
  But there lies the problem, a physical system is most often not linear.
  Also the estimated system as well as its variances over the time have to be known to provide this optimal estimation.
  If the noise matrices of the system are static, the filters gain matrices aim for a fix value and can therefore be calculated in beforehand.
  This reduces the computational effort by a significant amount. \cite{DavidWSchultz2004}.
  It should be mentioned that even if the noise is not Gaussian, the kalmanfilter is still the best
  linear estimator as long as the system and its properties are well known \cite{SimonDan2006Ose:}.
  
  \subsection{ROSE}
  The ROSE(rapid ongoing stochastic estimator) is in simple terms three kalman filters in one.
  Where the main filter is used as stated above, the additional two are used to estimated the 
  the system noise as well as the measuring noise. Therefore this sensor preforms better than the traditional kalman filter
  if those noises change over time in a not known fashion and has therefore be estimated.
  Due to this, this sensor needs more computational effort \cite{DavidWSchultz2004}. 
  
  \subsection{Extended Kalmanfilter}
  The extended kalmanfilter provides additional parts to better access nonlinearity in the observed system.
  This by not estimating the state of the system but by estimating the linearized change of the state 
  to the past sate. For this the systems equations have to be derived around the current working point in every estimation state.
  Therefore the computational effort exceeds even further \cite{SimonDan2006Ose:}.
  
  \subsection{Unscented Kalamnfilter}
  The unscented kalmanfilter 
  
  \subsection{H$\infty$ filter}
  The H$\infty$ filter is a more diverse approach then the ones described above.
  It was developed to access the problem when the to be observed system espacially its noise is not well known.
  Also in contrast to the kalman filters the H$\infty$ filter minimizes the worst-case estimation error 
  in stand of the standard deviation of the estimation error.
  
  \section{Choosing}
  If the requirements table \ref{tab:Requirements} is taken into the consideration of finding
  the optimal solution, two main requirements occur that define this desition.
  First the system load is a critical requirements and has therefore to be addressed in this process.
  Also for the requirement of modularity the algorithm should be as simple as possible.
  If taken in regard that the system is more or less well known and that the noise can be
  determened with the simulation and the log data from previous test flights,
  a normal kalmanfilter seems to be the most fitting solution.
  This because the performance of the rocket and the sensor should stay the same during
  each flight. 
  
  \section{System}
  The developed system or how you want to call it it doesn't really matter.
  will be hold simple to reduce the system load as well as prevent nonlinearities.
  Also the important values to estimate are at first hand the vertical height and speed, so
  for a first implementation just variables that can bed used do determine those both will be used.
  This are mainly the height and speed from the GPS, the vertical acceloration from the accelorometer
  as well as the pressure and temperature from the pressure sensors.
  
  
  describe here how the system desciption looks in perticular 
  \section{Explenation}
  How it 'should' work in detail
  
