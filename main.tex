\documentclass[a4paper, english, 10pt]{report}


%% Language and font encodings
\usepackage[english]{babel}
\usepackage[utf8x]{inputenc}
\usepackage[T1]{fontenc}

%% Sets page size and margins
\usepackage[a4paper,top=2cm,bottom=2.5cm,left=2.5cm,right=2cm,marginparwidth=1cm]{geometry}

%% Useful packages
\usepackage{amsmath}
\usepackage{graphicx}
\usepackage[colorinlistoftodos]{todonotes}
\usepackage[colorlinks=true, allcolors=black]{hyperref}

%% Own packages
\usepackage{subfiles}
\usepackage{fancyhdr}
\usepackage{fancyref}
\usepackage{listings}
\usepackage{subcaption}
\usepackage[toc,page]{appendix}
\usepackage{natbib}
\usepackage{pdfpages}

\setlength{\headheight}{30pt}
\bibliographystyle{plain}
\newcommand\NEVERRUNME{
    \bibliography{References.bib}
}

% Definings for inputcode in lisings
\usepackage{color}

\definecolor{mygreen}{rgb}{0,0.6,0}
\definecolor{mygray}{rgb}{0.5,0.5,0.5}
\definecolor{mymauve}{rgb}{0.58,0,0.82}
\definecolor{mypink}{rgb}{0,0.25,0.2}

\lstset{ %
	belowskip = 0pt,
  language = matlab,
  backgroundcolor=\color{white},   % choose the background color; you must add \usepackage{color} or \usepackage{xcolor}; should come as last argument
  basicstyle=\footnotesize,        % the size of the fonts that are used for the code
  breakatwhitespace=false,         % sets if automatic breaks should only happen at whitespace
  breaklines=true,                 % sets automatic line breaking
  captionpos=b,                    % sets the caption-position to bottom
  commentstyle=\color{mygreen},    % comment style
  deletekeywords={...},            % if you want to delete keywords from the given language
  escapeinside={\%*}{*)},          % if you want to add LaTeX within your code
  extendedchars=true,              % lets you use non-ASCII characters; for 8-bits encodings only, does not work with UTF-8
  frame=single,	                   % adds a frame around the code
  keepspaces=true,                 % keeps spaces in text, useful for keeping indentation of code (possibly needs columns=flexible)
  keywordstyle=\color{blue},       % keyword style
  morekeywords={pcl,PointCloud,::,Ptr,PointCloud2,line_t,line_p,lineRow_t,lineRow_p,landingField_p,landingField_t,PointXYZ},            % if you want to add more keywords to the set
  numbers=left,                    % where to put the line-numbers; possible values are (none, left, right)
  numbersep=5pt,                   % how far the line-numbers are from the code
  numberstyle=\tiny\color{mygray}, % the style that is used for the line-numbers
  rulecolor=\color{black},         % if not set, the frame-color may be changed on line-breaks within not-black text (e.g. comments (green here))
  showspaces=false,                % show spaces everywhere adding particular underscores; it overrides 'showstringspaces'
  showstringspaces=false,          % underline spaces within strings only
  showtabs=false,                  % show tabs within strings adding particular underscores
  stepnumber=2,                    % the step between two line-numbers. If it's 1, each line will be numbered
  stringstyle=\color{mymauve},     % string literal style
  tabsize=2,	                   % sets default tabsize to 2 spaces
  title=\lstname                   % show the filename of files included with \lstinputlisting; also try caption instead of title
}

% Fuss und Kopfzeile neu Definieren
\fancypagestyle{plain}{%
\fancyhf{} %alle Kopf- und Fußzeilenfelder bereinigen
\fancyfoot[R]{\thepage} %Seitennummer
%\fancyfoot[EL]{\thepage}
\fancyhead[L]{Bachelor Thesis}
\fancyhead[R]{Michael Kurmann}}

\pagestyle{plain}



% Title Page
\title{BAT - Sensor Fusion}
\author{Michael Kurmann}


\begin{document}
\begin{titlepage}
	\hbox{
			\hspace*{0.15\textwidth}
			\rule{1pt}{\textheight}
			\hspace*{0.05\textwidth}
			\parbox[b]{0.75\textwidth}{
{\scshape \large Lucerne University of Applied Science \& Architecture\par}
\vspace{0.2cm}
{\Huge \scshape \bfseries ARIS - Data Fusion for a Sounding Rocket\par}
\vspace{0.2cm}
{\scshape \large Bachelor Thesis\par}
\vspace{0.5cm}
\begin{center}
 \includegraphics[height = 10cm]{Pictures/ARIS_TELL_Badge.png}
\end{center}



%\vspace{0.5cm}
\begin{flushleft}
\begin{tabular}{l l}

Author: & Michael Kurmann \\
Department: & Electrical Engineering and Information Technology\\

Supervisor: & Prof. Marcel Joss\\

Expert: & Werner Scheidegger\\

Industrial Partner: & ARIS (Akademische Raumfahrt Initiative Schweiz) \\
		    & Oliver Kirchhoff\\

Submission date: & June 8, 2018\\
Classification: & Access
\end{tabular}
\end{flushleft}
\vspace{3cm}

}}

\vfill
\end{titlepage}


\chapter*{Declaration}
\thispagestyle{empty}
Hereby, I declare that I have composed the presented paper independently on my own and without any other resources than the ones indicated. All thoughts
taken directly or indirectly from sources are properly denoted as such.
This paper has neither been previously submitted to another authority nor has it been published yet.

\vspace{2cm}
Horw, June 8, 2018

\begin{abstract}
ARIS is a student association which competes every year in the SAC,
in which the aim is to build a rocket which can fly autonomous to an apogee of 10 000 feet.
For this a control loop is implemented which needs a an as accurate information as possible of the current height.
To achieve this different sensors are build into the rocket.
The aim of this thesis is to find an algorithm and implement a simulation in which the data of the sensors can be fused to estimate the height as good as possible.


To achieve this first a validation concept was developed. 
This concept uses a known trajectory to produce the perfect sensor data.
Those perfect sensor data are then adjusted to realistic sensor data which resemble the data which the real sensor do produce.
For this the logging data from test flights is used to produce AR models which are used to generate the noise for the realistic data.
After that this produced realistic sensor data is put into different versions of state estimator,
The estimated heights from those state estimator are then compared to the trajectory which was used at the beginning.


As the best fitting fusion algorithm a discrete Kalmanfilter was found.
It works with dynamic noise matrices to compensate for different sampling speeds of the sensors.
The Kalmanfilter itself does work with an rank 5 system which consists of height,speed,acceleration,acceleration offset and the pressure.
This solution provides a in perspective simple algorithm which is with the help of the simulation easy to adjust.
This makes sure that the algorithm will work on a embedded system an can be used again in coming competition with appropriated effort.
\end{abstract}
\tableofcontents

\chapter*{Shortcuts and Variables}
\section*{Shortcuts}
\begin{tabbing}
 ARIS    \hspace{5cm} \= Akademische Raumfahrt Initiative Schweiz \\
 IREC 		\> Intercollegiate Rocket Engineering Competition \\
 SAC		\> Spaceport America Cup \\
 ERT		\> EPFL Rocket Team \\
 EPFL	  	\> École Polytechnique Fédérale de Lausanne \\
 GPS 		\> Global positioning system \\
 FIR filter 	\> Finite impulse response filter\\
 IIR filter 	\> Infinite impulse response filter\\
 AR model 	\> Auto Regressive model\\
 ARMA model 	\> Auto Regressive Moving Average model \\


\end{tabbing}
\section*{Variables}
\begin{tabbing}
 A \hspace{5.55cm}	\= System matrix \\
 Ad 		\> Discrete system matrix \\
 B 		\> Input matrix \\
 Bd 		\> Discrete input matrix \\
 C 		\> Output matrix \\
 D 		\> Throughput matrix \\
 x		\> State vector \\
 u		\> Input vector \\
 y		\> Output vector \\
 G 		\> System noise input matrix \\
 Gd 		\> Discrete system noise input matrix \\
 Q 		\> System noise matrix \\
 R 		\> Measurement noise matrix \\
 K 		\> Kalman gain matrix \\
 P 		\> Error covariance matrix \\ 
 P0 		\> Pressure at ground level \\
 T0 		\> Temperature at ground level \\
 Tgrad 		\> Temperature gradient for the actual weather condition \\
 M 		\> Molar mass of Earth's air: 0.0289644 kg/mol\\
 g 		\> Gravitational acceleration: 9.80665 m/$s^2$\\
 R 		\> Universal gas constant: 8.3144598 J/mol/K\\
 $a_1 \hdots a_M$ \> FIR filter coefficients  \\
 $\sigma_{\omega}$ \> Standard deviation of noise $\omega$ \\
 $\gamma_{yy} $ \> Autocorrelation of y \\
 h		\> Height \\
 v 		\> Speed \\
 a 	 	\> Acceleration \\
 p		\> Pressure \\
 $\varphi$	\> Pitch angle \\
 $KP_v$		\> Pressure gain depending on speed
\end{tabbing}

\chapter{Introduction}
\label{ch:Introduction}
 \documentclass[main.tex]{subfiles}


 \cite{SimonDan2006Ose:}

 
 \section{Bla}
 About Sensor Fusion and ARIS
 
 
 \section{Research}
 Write about the Papers/book you used:
 Kalman-filter
 Optimal state estimation
 The Master Thesis
 
 \section{Problem}
 State the problem, what will be difficult ? For what is it ? where should it be improved
 
 Problems found so far:
 \begin{itemize}
  \item How to calculate Height out of Pressure/Temp/Humidity Fabian ver: $44330 * (1 - (\frac{pressure}{101325})^{ \frac{1}{5.255}})$
  \item How to integrate AirBreaks/Drag Force of Air/ Trust of Motor ?
  \item The up-flight is rather short: about 25s so the Fusion should have a small settling time
  \item The Chip on which it is used is no the fastes : 168 MHz clock
  \item The Ram on the Chip is not endless: Maximal space for the Sensor fusion is about 10kB
  \item The Sensor Fusion should be as modular as possible so that it also can be used in the next competition
  \item The Sensor Fusion has to be as sturdy as possible so that it will not fail if a problem occurs
  \item The Fusion should make a state Estimation as percicse as possible.
  \item There are a lot of different variables: 3 Positions, 1 Speed, 3 Accelorations, 3 Lagen, Time, Pressure, Tempterature, Humidity, Up-/Downforce.
  \item Especially the Input Value u which is the force onto the rocket is difficult to define (Drag, Trust = acceloration depends on wheigt which changes over time).
  \item The different Sensor have different weaknesses: \begin{itemize}
							 \item Accelometer: Offset, drift, weak to vibrations
                                                         \item Gyro: Weak to Vibrations
                                                         \item Barometer: Many uncertenties, unpercise
                                                         \item GPS: Slow (max 5Hz)
                                                        \end{itemize}
                                                        
								  
 \end{itemize}

 
 
\section{Requirements}
These are the requirements which were drown out of the problems list.
 
 \begin{table}[h]
\centering
\begin{tabular}{|l|l|l|l|}
\hline
Requirement   & \multicolumn{1}{c|}{Rating} & Aim & Importance \\ \hline
Precision     & Error between estimation and groundtruth  & < 5\% after settling time & High  \\ \hline
Sturdiness    & \% Percision without \# sensors  & -10\% per failed sensor  & Medium \\ \hline
Code Size     & kB of RAM needed  & < 10kB size & High \\ \hline
System Load   & Time per loop on the cuurent Chip (128 MHz) & < 1ms & Crictal \\ \hline
Non Linearity & \# non linearities in the algortihm  & 0 &  Desirable\\ \hline
Settling time & Time from ignition to optimal estimation  & < 5s       &  Critical\\ \hline
Modularity    & Effort needed to integrate a new sensor    & < 10h work     &  Desirable\\ \hline
              &                            &            &  \\ \hline
              &                            &            &  \\ \hline
              &                            &            &  \\ \hline
\end{tabular}
\caption{Requirements table}
\label{tab:Requirements}
\end{table}
 
 \section{Desired Solution}
 
 Describe in quick terms what you are aiming for.

\chapter{Approach}
\label{ch:Approach}

  How I want to get to the Solution and why I choosed it also how it works.
  
  \section{Tipps}
  - Don't forget point of origin / reference system
  - Don't forget the pitch angle
 
  
  \subsection{Barometer}
  -Use density as a statevector \\
  -Use exponential atmosphere method. \\
  -Increasing the R measurement noise matrix when rocket is ascending access the rising uncertainties
  
  
  \section{Possibilities}
  Which ones are there and what are their positives and negatives, in short how do they work
  I will use kalman filter with time depending system and sensor noise, but i have yet to define how i do this in peticullar.
  I should also make some pictures and such things.
  
  \section{Choosing}
  Why i choosed Kalman Filter
  
  \section{Explenation}
  How it 'should' work in detail
  


\chapter{Implementation}
\label{ch:Implementation}
This chapter describes how the simulation is implemented in detail.
The development of the simlation has been done in Matlab.
The first part of the chapter mainly discusses the sensor model implementations while the second part is about the implementation of the state estimator itself.

\section{Sensor Models}
How the concept of the different sensor models work is described in chapter \ref{ch:Approach}.
In the following the implementation which is used in the simulation will be described in detail.
First in general for all sensors, followed by the different characteristics of each sensor.

\subsection{Perfect Sensor}
To calculate the prefect sensor data the trajectory is put into the equations which were defined before.
In Figure \ref{fig:GeneratedPerfectSensor} those generated sensor data as well as the trajectory used for this can be seen.

\begin{figure}[h!]
 \centering
 \includegraphics[width=0.8\textwidth]{./Pictures/GeneratedSensorData.jpg}
 % GeneratedSensorData.jpg: 0x0 pixel, 300dpi, 0.00x0.00 cm, bb=
 \caption{Generated sensor data}
 \label{fig:GeneratedPerfectSensor}
\end{figure}

\subsubsection{Accelorometer}
Due to the fact that the whole simulation works with discrete time stamps the derivative of the height can not be done formally.
So it is done by calculating the difference between each data point to the next and then weighting those by the delta in time between them.
This has to be done two times to get from the height to the acceleration.
The unit for the acceleration in this simulation is meters per second squared.

\subsubsection{Gyrometer}
As stated before the pitch angle can not be directly generated.
But as can be seen from the data of the test flights the angle stays more or less the same wile the motor is burning.
This make sense because during this time the main acceleration comes from one determined direction and stabilises the rocket.
After the burnout the pitch angle does change more or less randomly depending on strength and direction of the wind that hits the rocket.
To simulate this the values are generated randomly and then low pass filtered with a moving average filter to represent that behaviour.
While doing this the random values are kept small during the burning of the motor and raised afterwards to higher values.

\subsubsection{Barometer}
The measurements from the barometers are depending on the atmospheric model which is used in this simulation.
For reasons of simplicity the start pressure is chosen as the mean pressure at sea level which is 1013.35 hP.
Also the temperature at the beginning is chosen as 288.15 Kelvin (15 °C) which also represents the mean value on the sea level.
Last but not least the temperature gradient is chosen as - 0.0065 K/m, which is a commonly used value.
For the state estimation in a test flight those values have to be determined before the start, especially the temperature gradient.

Since the barometers do sample at 100 Hz the measurements have to be down sampled.
As with the GPS sensor below this is achieved by a zero order hold conversion instead of direct down sampling

\subsubsection{GPS}
As stated in chapter \ref{ch:Approach} the GPS signal is just the height with a different sampling time.
To maintain the vectors length which simplifies the later use in the estimation algorithm,
the signal is acquired with a zero order hold conversion instead of a down sampling.

%@ToDo: Add code for the zero order hold convertion

\subsection{Noise}
To generate the noise out of the data from the test flight, it has first to be extracted.
It is assumed that the noise of the data is different depending on the state of the rocket (before ignition, during motor burning, after burnout untill parachute ejection),
but it should have more or less the same properties between those events.
Based on this, the data vector has to be separated in those different sections first.
For this the accelerometer measurements are iterated to find the time stamps on which those events happen, as can be seen in Figure \ref{fig:AccelerationMarks}.


\begin{figure}[h!]
 \centering
 \includegraphics[width=0.8\textwidth]{./Pictures/AccelerationMarks.jpg}
 % AccelerationMarks.pdf: 0x0 pixel, 300dpi, 0.00x0.00 cm, bb=
 \caption{Timestamps drawn out of acceleration measurements}
 \label{fig:AccelerationMarks}
\end{figure}


If done so, polynomials of the order two are fitted on this measurements with the least squared error method.
Those polynominals represent the functions / the values which are assumed to be the noiseless data with possible offsets.
So if now these polynominals are subtracted from the test flight data the result are the measurements with zero mean.
From this point on this noise can be examined on its parameters, like the power density, the probability distribution and the variance, as can be seen in Figure \ref{fig:PF_AC_HIST_Accel}.

%% pictue of autocorrelation, histogramm etc from sensor data noise
\begin{figure}[h!]
 \centering
 \includegraphics[width=0.8\textwidth]{./Pictures/PF_AC_HIST_Accel.jpg}
 % PF_AC_HIST_Accel.jpg: 0x0 pixel, 300dpi, 0.00x0.00 cm, bb=
 \caption{Polyfit Autoccorellation and Histogramm}
 \label{fig:PF_AC_HIST_Accel}
\end{figure}


This noise data can now be used to solve the Yule Walker Equation to get an AR model.
For this the aryule() function can be used which estimates an AR model of the order N as well as the variance directly out of the noise vector.
To achieve this it estimates the autocorrelation out of the given noise vector.
For the best possible results the corresponding data from all test flights were put together and used as one great noise vector.
Before doing this, the data had to be resampled so that the AR models can be properly used in the simulation.
With those AR models, the noise can be regenerated by filtering white noise with the correct variance.
This generated noise can now be compared to the noise from the test flight data.

%% picture of pwelch plot from both noise vectors
\begin{figure}[h!]
 \centering
 \includegraphics[width=0.8\textwidth]{./Pictures/PDSnoise.jpg}
 % PDSnoise.jpg: 0x0 pixel, 300dpi, 0.00x0.00 cm, bb=
 \caption{PDS from the measured and generated noise}
 \label{fig:PDSNoise}
\end{figure}


As can be seen in figure \ref{fig:PDSNoise} both noises resemble each other in their power density spectrum much more than the white noise would.
So this AR model is exported to the simulation script and can be used there to generate the real sensor data.

\subsubsection{Accelorometer}
The noise which is on the accelorometer is special because it often has a drift which results in a more or less constant offset.
To recreate this, the offset can be estimated from the test flight data.
Especially the data before the ignition is helpful, because the value that should be measured is known.

\subsubsection{Gyrometer}
For the gyrometer noise a separate script has been written to calculate the proper pitch angle and filter out the offset before generating the AR model.
This because the gyrometer measurements which are available are in degrees per second
and have therefore to be integrated before they resemble the correct pitch angle.
In addition to this it is complicated to define which part of the measurements are noise and which is the ground truth.
It was found that the estimated AR model could not regenerate noise with the same characteristics.
Because of that, the noise has to be low pass filtered another time to resemble the real noises better.
For this task an IIR filter of order 200 was found best.
With this also the Gyrometer measurements could be properly handled.

\subsubsection{Barometer}
The barometer noise itself has no real capacities which were not already discussed before.

\subsubsection{GPS}
The GPS noise capacities were only found with measurements that were taken for a longer time period while the GPS receiver stayed at the same place.
So based on the assumption that the GPS measurements are independent from the motor vibrations and the rocket posture changes the noise will have the same capacity over the whole flight.
This should be suitable as long as the receiver does not lose its fix.

\newpage
\subsection{Real Sensor}
To now generate the real sensor data, the different noises have to be generated with the calculated AR models and added to the perfect sensor data.
For this a vector of normal distributed random values is generated and multiplied by the square root of the corresponding variance (hence the standard deviation).
This white noise is now filtered by the corresponding AR-model and can then be added onto the corresponding perfect sensor data, which results in the real sensor data.

\subsubsection{Accelorometer}
\begin{figure}[h!]
 \centering
 \includegraphics[width=0.8\textwidth]{./Pictures/AccelPerfVSReal.jpg}
 % AccelPerfVSReal.jpg: 0x0 pixel, 300dpi, 0.00x0.00 cm, bb=
 \caption{Plot of perfect acceleration vs realistic vs measured}
 \label{fig:AccelPerfVsReal}
\end{figure}
In Figure \ref{fig:AccelPerfVsReal} the different generated values as well as the data from a test flight can be seen.
The time vector from the test flight was stretched by the factor two to make the observation easier.
Also it has to be said that the test flight was with a smaller rocket which flew only at an apogee of around 300 meters.
This explains why the acceleration is not as great as in the generated data and why the time vector had to be stretched.
But the plot shows that the noise as well as the perfect data resemble the acceleration from the test flight in a appropriate way.

\newpage
\subsubsection{Gyrometer}
\begin{figure}[h!]
 \centering
 \includegraphics[width=0.8\textwidth]{./Pictures/PitchPerfVSReal.jpg}
 % PitchPerfVSReal.jpg: 0x0 pixel, 300dpi, 0.00x0.00 cm, bb=
 \caption{Plot of perfect gyrometer vs realistic vs measured}
 \label{fig:PtichPerVSReal}
\end{figure}
Figure \ref{fig:PtichPerVSReal} shows the generated gyrometer data.
Like in the acceleration plot the time vector from the test flight data was adjusted for better observability.
It should also be explained that due to the property of the pitch angle (more or less random depending on air current etc)
the generated realistic pitch angle must not resemble the measured angle in its specific value.
Important to show is that the noises have the same capacities which they do.
Also it can be seen that the assumption that the angle does not change much during the burning of the motor is appropriate despite a quick change at the start.

\newpage
\subsubsection{Barometer}
\begin{figure}[h!]
 \centering
 \includegraphics[width=0.8\textwidth]{./Pictures/PressurePerfVSReal.jpg}
 % PressurePerfVSReal.jpg: 0x0 pixel, 300dpi, 0.00x0.00 cm, bb=
 \caption{Plot of perfect barometer vs realistic vs measured}
 \label{fig:PressurePerfVSReal}
\end{figure}
The realistic pressure measurements from a barometer are shown in Figure \ref{fig:PressurePerfVSReal}.
The noise itself does not look as it would have a great impact on the perfect data.
But because the pressure does change by around 350 hP during the upflight and the noise is only around 1 to 3 hP it can not be seen that good.
The comparison with the real measured data (in the figure also with a stretched time vector) shows that the generated realistic measurement data does resemble the real measurements well.

\newpage
\subsubsection{GPS}
\begin{figure}[h!]
 \centering
 \includegraphics[width=0.8\textwidth]{./Pictures/GPSPerfVSReal.jpg}
 % GPSPerfVSReal.jpg: 0x0 pixel, 300dpi, 0.00x0.00 cm, bb=
 \caption{Plot of perfect GPS vs realistic vs measured}
 \label{fig:GPSPerfVSReal}
\end{figure}
Last but not least the generated GPS measurements can be observed in Figure \ref{fig:GPSPerfVSReal}.
In this plot the pure generated noise was also plotted because this way it can be best shown that it has real slow properties.
For comparison the measurements which were taken from a fixed position are plotted (because there were no usable GPS data from a test flight avaiable during the term of this thesis).
It can be seen that the test data does also resemble the noise from the generated data.


\section{State Estimation}
As stated in the last chapter the to be used state estimator is a discrete Kalman filter with dynamic noises on the measurements as well as on the system.
For this the noise matrices as well as the used loop is described below.

\subsection{System Model}
Based on the models stated in chapter \ref{ch:Approach} several different implementations can be derived.
For this simulation 8 models were implemented with each different capabilities.
To find the best suiting model all of them have been tested. The test results are discussed and evaluated in the next chapter.
\subsection{Adjustment}
Taking into account the noise characteristics of the measurements stated above, some adaptions to the model presented in chapter \ref{ch:Approach} have been made, which should hopefully result in better results.
\subsubsection{Offset}
The main adjustment is the inclusion of acceleration offset into the state vector.
This is a common tactic to minimise the impact of the offset. This works because by adding the offset to the state vector the state estimator can also estimate the actual offset which can then be used to correct the offset value \cite{DavidWSchultz2004}.
Therefore the state vector of a point mass would look like this.
$$ x = \begin{bmatrix}
        h_z \\
        v_u \\
        a_z \\
        a_{offset} \\
       \end{bmatrix}
$$
While the dynamic matrices A and B would stay the same apart from an additional dimension of zeros for the acceleration offset state variable.
\begin{align*}
 A = \begin{bmatrix}
      0 & 1 & 0 & 0 \\
      0 & 0 & 1 & 0 \\
      0 & 0 & 0 & 0 \\
      0 & 0 & 0 & 0
     \end{bmatrix}
     & \hspace{1cm}
 B = \begin{bmatrix}
      0 \\
      0 \\
      0 \\
      0
     \end{bmatrix}
\end{align*}
The y vector would stay the same while the output matrix $C^T$ will be adjusted so that both acceleration and acceleration offset in the state vector are added into the accelerometer measurements.
\begin{align*}
 y = \begin{bmatrix}
      h_{GPS} \\
      h_{p1} \\
      h_{p2} \\
      a
     \end{bmatrix}
     & \hspace{1cm}
 C^T = \begin{bmatrix}
      1 & 0 & 0 & 0 \\
      1 & 0 & 0 & 0 \\
      1 & 0 & 0 & 0 \\
      0 & 0 & 1 & 1
     \end{bmatrix}
\end{align*}

\subsubsection{Acceleration as Input}
An additional adjustment would be to consider the measured acceleration of the rocket as input into the system.
This should result into a system which could react faster to changes in the acceleration.
For this the measurement noise of the accelerometer would have to be placed in the system noise matrix
and therefore no additional system noise can be modulated.
For a point mass this would result in the following system matrices.
While the state vector and the A matrix would stay the same, the B vector would have to be adjusted like this.
$$ B = \begin{bmatrix}
        0 \\
        0 \\
        1
       \end{bmatrix}
$$
In addition the y vector would lose its acceleration measurements and the output matrix $C^T$ the corresponding dependencies.
\begin{align*}
 y = \begin{bmatrix}
      h_{GPS} \\
      h_{p1} \\
      h_{p2} \\
     \end{bmatrix}
      & \hspace{1cm}
 C^T = \begin{bmatrix}
        1 & 0 & 0 \\
        1 & 0 & 0 \\
        1 & 0 & 0 \\
       \end{bmatrix}
\end{align*}

\subsection{Measurement noise}
The matrix on each time stamp for the measurement noise is rather easy to get in the simulation because the perfect measurements are known.

First the variance over the burning of the motor as well as over the up flight is calculated separately.
After that those values are used to generate a noise vector for each measurement.
In addition the implementation of the noise vector has the same length as all other used vectors.
These noise measurements are then concatenated into diagonal noise matrices as shown in the listing below:

\begin{lstlisting}[caption={Mesurements noise matrix conagonating}]
R_dyn = [GPSvar;ACLvar;HBM1var;HBM2var];         %Add the noise vectors into one matrix
R_dyn_m = diag(R_dyn(:,1)');                     %Make a diagonal first diagonal matrix
for n = 2:length(TimeVec)
    R_dyn_m = cat(3,R_dyn_m,diag(R_dyn(:,n)'));  %Concatogenate all noise vectors
end
\end{lstlisting}


If the noise matrix is displayed as a diagonal matrix,
it takes into account the assumption that the noises from the measurements are independent from each other.
This assumption can be made because of the fact that each measurement except those from the barometer (pressure and temperature) are made from different sensors.
Due to the fact that the temperature is not used for the state estimation, the measurements matrix can still be assumed as diagonal.

\subsubsection{Different sampling times}
In addition the measurement noise matrix can be used to adjust for the different sampling times of the sensors.
This is used for the barometers as well as the GPS sensors which are sampled slower as the state estimation itself loops.
This adjustment is achieved by maxing out (setting to the highest possible value) the corresponding variance in the measurement noise matrix R if no actual measurements are available.
As it can be seen in the formula to calculate the Kalman gain K,
$$  K = P\cdot C^T\cdot (C\cdot P\cdot C^T + R)^{-1} $$
maximum values in the R matrix result in nearly zero values in the corresponding K matrix.
Those near or exactly zero values result in ignoring the corresponding measurements from the y vector as can be seen in the measurement update equation.
$$  x = \hat{x} + K\cdot(y[k] - C^T \cdot \hat{x}) $$

With this vectors with the same length as the other vectors in the state estimation can be generated out of the measurements.
When implementing in an embedded system this can simply be achieved with an ``if'' statement that switches the R matrix value to the normal variance if a measurement arrives.

\subsection{System noise}
The system noise describes how uncertain the system model is in comparison to the real system.
For this each entry in the diagonal matrix resembles the variance of the noise on the corresponding state variable.
In other words the system noise describes how far away from the predicted value the actual value can get in the next loop iteration.
For the system noise the behaviour of the system during the flight has to be examined.
This can be done in different ways.
The first way would be to view the different ground truth curves of those state variables,
which do have system noise acting on them (acceleration, pitch angle, pressure if linearised).
\begin{figure}[h]
 \centering
 \includegraphics[width=.8\textwidth]{./Pictures/PitchAnglePlot.jpg}
 % PitchAnglePlot.jpg: 0x0 pixel, 300dpi, 0.00x0.00 cm, bb=
 \caption{Plot of pitch angle}
 \label{fig:PitchAnglePlot}
\end{figure}
For example Figure \ref{fig:PitchAnglePlot} shows the pitch angle.
In the system models there are no influences on this state value except for the measurements.
So to describe the changes of the value which are observed with the measurements a system noise has to have an impact on the pitch angle.
As it can be seen in the plot the angle does not change in a great manner until the burnout so the system noise till the burnout should also be rather small.
After the burnout the angle changes a lot more but keeps changing in the same way all over the time. \\

As a second attempt since the system noise describes the capability of the value to change over time independent from the dynamic system description
it can also be achieved by deviating the ground truth value vector.
For this the following matlab code can be used.
\begin{lstlisting}[caption={System noise generation with deviation}]
ACEL = abs(diff(a));                            % Deviating the gournd truth acceleration
ACEL = filter(ones(1,100)*1/100,1,ACEL);        % Low pass filtering
ACEL = [ACEL ACEL(end)];                        % Maintain vector length
\end{lstlisting}
This shows that after the deviation the absolute value from the deviation result is taken since a variance cannot or should not be negative.
After that the values are low pass filtered to make a more steady system noise description.

\subsection{Sensor Outfall}

An additional interesting scenario which can be observed with the simulation is the outfall of sensors.
This is needed to test the reliability requirements which states that the algorithm should still be working if 2-3 sensor fail, accepting that the estimation results will not be as accurate any more.
For this it has to be said that it has to be detected that a sensor fails to adjust the estimation algorithm.
If done so the variance can be adjusted for this sensor in the same way as stated above for the GPS signal by maxing its variance out.

It is achieved by a simple if statement which does exactly that if a sensor fail is recognised.

\subsection{Loop}
Finally the state estimation is implemented in a simple loop which iterates trough each given time stamp.
First the needed vectors and matrices have to be initialised with the right value.
In most system model versions the u vector remains zero while all measurements are brought into the estimation loop trough the y vector.
But there are some others where the acceleration and the pitch angle are brought into the estimation loop over the u vector only and the remaining measurements through the y vector.
That means that the current state vector x has to be initialised with the value of the corresponding sensors at the start,
which is presumably zero for all states except pressure and temperature.
The loop itself calculates the equations as they were stated in chapter \ref{ch:Approach}.

In addition if values from the measurements can not be transformed into the state vector values directly or only with a linearising calculation,
they have to be transformed first before entering the system.
For example pressure and temperature into height or acceleration and pitch angle into pure vertical acceleration.\\

Below is an example for the estimation loop for a rank five system. This model contains height, speed, acceleration, acceleration offset and pitch angle as state variables.
\begin{lstlisting}[caption={State Estimation Loop}]
% Initalzation
u = zeros(1,length(TimeVec));                       %Input vector is zero
y = [h_mes_GPS;a_mes;p_mes_1;p_mes_2;phi_mes];      %Output are the measurements
x = [0;0;0;0;0];                                    %Start Vector should be like this
P = eye(5);                                         %Standart can maybe be increased
Height1 = 0;
Height2 = 0;
Temp = T(1);

% Estimation loop
x_est_loop = zeros(size(x,1),length(TimeVec));     %Vector for the SE values
for k = 1:length(TimeVec)
    K = P*C'*pinv(C*P*C' + R_dyn_m(:,:,k));
    Height1 = CalcHeight(Po,p_mes_1(k),Temp,0,true,TgradSimu);
    Height2 = CalcHeight(Po,p_mes_2(k),Temp,0,true,TgradSimu);
    acc = a_mes(k) * cos(x(5)*pi/180);
    x = x + K*([h_mes_GPS(k);acc;Height1;Height2;phi_mes(k)] - C*x);
    P = (eye(5)-K*C)*P;

    x_est_loop(:,k) = x;                           %Save data from the Sensor fusion

    x = Ad*x + Bd*u(k);
    P = Ad*P*Ad' + Gd*Q_dyn_m(:,:,k)*Gd';

end
\end{lstlisting}



\chapter{Tests}
\label{ch:Tests}
In this chapter the test results of the tests that were taken to gain insight of the different system models and their performances are discussed.
First a general analysis is made by plotting the whole estimated flight as well a table which displays the errors of the different estimated values which were taken from multiple simulation cycles.
For this all estimations have been made with a mean acceleration offset of $ -1.31 m/s^2$  which was taken out of measurements from a test flight.
Also for the general analysis a perfectly linear temperature gradient of $-0.0065 °/m$ was used.
After that the different problematic properties of each system model is discussed in detail.
At the end the best system from the first tests has been further tested under different circumstances.


\section{Point Mass}
The first system model to test was the simple point mass model as described in chapter \ref{ch:Approach}.
While this implementation is a simple version it does already work surprisingly good.
In Figure \ref{fig:PointMassPerformance} the overall performance of its estimation of the different values can be seen.


\begin{figure}[h!]
 \centering
 \includegraphics[width=.8\textwidth]{./Pictures/PointMassPerformance.jpg}
 % PointMassPerformance.jpg: 0x0 pixel, 300dpi, 0.00x0.00 cm, bb=
 \caption{The performance of the point mass system model over time}
 \label{fig:PointMassPerformance}
\end{figure}

Especially the first second (lower right corner) and the last five seconds (upper right corner) are interesting.
It can be seen that the system does not really have a settling time at all,
because the height is estimated at the minimal error at the beginning of the estimation.
On the other hand the speed needs around 0.2 seconds to settle on the right value.
Also a clear error on the estimation of the error can be seen after the speed has settled,
this is due to the fact that the estimator can trust the GPS and pressure measurements much more at the beginning than the acceleration measurements.
Therefore the estimator does adapt the acceleration estimation on the height and speed estimation at the start and not the other way around.
In the last five second it can be seen that the height is far more off as it was at the start.
Also the steps the estimation makes there are coming from the GPS measurements.
In addition Table \ref{tab:ErrorPointMass} shows the maximum, minimum, mean and median error of the estimated values.

\begin{table}[h!]
\centering
\begin{tabular}{cccccc}
\hline
\multicolumn{1}{|c|}{State Variable} & \multicolumn{1}{c|}{Unit} & \multicolumn{1}{c|}{Max} & \multicolumn{1}{c|}{Min} & \multicolumn{1}{c|}{Mean} & \multicolumn{1}{c|}{Median} \\ \hline
Height                            & $m$                         & 20.05                  & 1.18e-05                 & 4.72                    & 2.20                      \\
Speed                             & $m/s$                       & 3.42e+03               & 0                        & 3.40                    & 2.43                      \\
Acceleration                       & $m/s^2$   			& 17.79                  & 2.55e-05                 & 1.67                    & 1.55
\end{tabular}
\caption{Error of estimated state variables}
\label{tab:ErrorPointMass}
\end{table}

The most interesting value is the median from the height error because it is free from outliers.
It shows that the height error is with 2.2 meter slightly above the aimed 2 meter.
The 20 meter maximum error occurs around second 15 of the flight just before a new GPS measurement comes in.
At this time stamp the offset on the acceleration measurements does have a greater impact because the real value is in comparison to the start rather small.
In addition the height does change more between two GPS measurements than it does at the end of the flight, therefore greater errors in the interpolation between can occur.
Additionaly the minimum speed error occurs at the beginning of the estimation loop where the speed still zero.


\subsection{Greater Offset}
It has to be said that this only works while no sensor has a bigger offset.
Therefore these values come at the cost that they are not that trustworthy,
because the system noise on the acceleration has to be set to a greater value to get those good estimations.
An additional factor is also the pitch angle which does hinder this estimation if it changes in a great manner.
This can be seen in Figure \ref{fig:PointMassErrorWithOffset} which shows the state estimation error with different offsets.

\begin{figure}[h!]
 \centering
 \includegraphics[width=.8\textwidth]{./Pictures/PointMassErrorWithOffset.jpg}
 % PointMassErrorWithOffset.jpg: 0x0 pixel, 300dpi, 0.00x0.00 cm, bb=
 \caption{Error during flight time with different offsets}
 \label{fig:PointMassErrorWithOffset}
\end{figure}


Table \ref{tab:PointMassPerformanceWithOffset} shows the mean and median of the error in the height depending on the offset on the accelerometer.

\begin{table}[h!]
\centering
\begin{tabular}{ccc}
\hline
\multicolumn{1}{|c|}{Offset} & \multicolumn{1}{|c|}{Mean}& \multicolumn{1}{|c|}{Median} \\ \hline
%
% & Mean & Median\\
Normal & 4.57 & 2.80\\
2 Times & 13.97 & 14.66\\
4 Times & 39.31 & 46.99\\
6 Times & 59.35 & 71.90\\
8 Times & 107.21 & 102.15
\end{tabular}
\caption{Error of the height in meter with changing offset}
\label{tab:PointMassPerformanceWithOffset}
\end{table}

This shows that the error which the estimator makes does rise exponentially.
So this is an issue which should be assessed by including an acceleration offset in the state vector.


\newpage
\section{Point Mass with Acceleration Offset}
The overall performance of this system model can be seen in the figure \ref{fig:PointMassOffsetPerformance}.

\begin{figure}[h!]
 \centering
 \includegraphics[width=.8 \textwidth]{./Pictures/PointMassOffsetPerformance.jpg}
 % PointMassOffsetPerformance.jpg: 0x0 pixel, 300dpi, 0.00x0.00 cm, bb=
 \caption{Performance of point mass with acceleration offset over time}
 \label{fig:PointMassOffsetPerformance}
\end{figure}

While the speed and height of it performs more or less the same way at the first second like the point mass system model,
a clear difference in the estimated acceleration can be seen.
This is possible due to the fact that the estimator can describe the difference in the system description between the speed and acceleration as the acceleration offset.
These dependencies can be seen after the settling of the speed where the estimator starts to change the acceleration offset value to adjust the acceleration.
The great advantage of this can be seen in the plot of the last five seconds of the height estimation (upper right plot).
Due to the better estimation of the acceleration over the whole time, the dependencies of the height on the acceleration is much more trustworthy.
When the rocket does rise at further height and the measurements of the barometers lose their accuracy,
the better acceleration measurements can be used to interpolate between the good GPS measurements.
This results in a much better height estimation at greater height.

\begin{table}[h!]
\centering
\begin{tabular}{cccccc}
\hline
\multicolumn{1}{|c|}{State Variable} & \multicolumn{1}{c|}{Unit} & \multicolumn{1}{c|}{Max} & \multicolumn{1}{c|}{Min} & \multicolumn{1}{c|}{Mean} & \multicolumn{1}{c|}{Median} \\ \hline
Height                            & $m$                         & 9.78                   & 2.98e-06                 & 1.33                    & 0.83                      \\
Speed                             & $m/s$                       & 78.79                  & 0                        & 3.18                    & 2.22                      \\
Acceleration                       & $m/s^2$   			& 165.87                  & 1.35e-05                 & 4.52                    & 2.64                     \\
Acceleration Offset                & $m/s^2$   			& 167.46                  & 2.66e-05                 & 4.44                    & 2.61
\end{tabular}
\caption{Error of estimated state variables point mass with acceleration offset}
\label{tab:ErrorPointMassAccelerationOffset}
\end{table}

Table \ref{tab:ErrorPointMassAccelerationOffset} shows the estimation errors of this model.
The entries that draw the attention are the maximum of the estimation errors from the acceleration and speed which is bigger than that of of the point mass system model.
These occurred due to one complete false estimation during one of the simulation and should not be seen as normal.
The interesting thing about this is that despite those big maxima, the mean value of the estimated height and speed are still better than the ones from the point mass system model.
Also can be seen that the median of the height error is just 0.83 meter. \\

As with the point mass system model Figure \ref{fig:PointMassOffsetErrorWithOffset} shows the error during a flight with different sensor offsets.
It can clearly be seen that while the mean error does rise by some value, it always gets back to zero and therefore overall this system preforms better as the simple point mass.

\begin{figure}[h!]
 \centering
 \includegraphics[width=0.8\textwidth]{./Pictures/PointMassOffsetErrorWithOffset.jpg}
 % PointMassOffsetErrorWithOffset.jpg: 0x0 pixel, 300dpi, 0.00x0.00 cm, bb=
  \caption{Error during flight time with different offsets}
 \label{fig:PointMassOffsetErrorWithOffset}
\end{figure}


\newpage
\section{Point Mass with Pressure}
As with the ones above the overall performance of the system model which uses the pressure as an additional state variable can be seen in Figure \ref{fig:PointMassPressurePerformance}.

\begin{figure}[h!]
 \centering
 \includegraphics[width=.8 \textwidth]{./Pictures/PointMassPressurePerformance.jpg}
 % PointMassOffsetPerformance.jpg: 0x0 pixel, 300dpi, 0.00x0.00 cm, bb=
 \caption{Performance of point mass with pressure over time}
 \label{fig:PointMassPressurePerformance}
\end{figure}

This shows that it performs as good as a system which uses acceleration offset as a state value.
It can be seen that the pressure is so well estimated that it fully covers the real pressure curve in the plot.
Also in the plot of the first second it can be seen that there is no settling time for either of the estimation values.
The height in the last 5 second shows too that the impact from the GPS measurements are much smaller than in the first model (no staircase like adjustments of the height as seen in the point mass system model).

\begin{table}[h!]
\centering
\begin{tabular}{cccccc}
\hline
\multicolumn{1}{|c|}{State Variable} & \multicolumn{1}{c|}{Unit} & \multicolumn{1}{c|}{Max} & \multicolumn{1}{c|}{Min} & \multicolumn{1}{c|}{Mean} & \multicolumn{1}{c|}{Median} \\ \hline
Height                            & $m$                         & 5.16                   & 1.47e-06                 & 1.20                    & 0.96                      \\
Speed                             & $m/s$                       & 8.80                   & 0                        & 1.89                    & 1.61                      \\
Acceleration                      & $m/s^2$   			& 18.14                  & 3.90e-05                 & 1.80                    & 1.63                      \\
Pressure                  	  & $hPa$   			& 0.72                   & 4.69e-06                 & 0.13                    & 0.11
\end{tabular}
\caption{Error of estimated state variables point mass with pressure}
\label{tab:ErrorPointMassPressure}
\end{table}

Like above Table \ref{tab:ErrorPointMassPressure} shows the error over the simulations.
The error on the height is nearly as good as with the acceleration offset, while the maxima of the differences are quite smaller than above.
These results therefore have much better mean values of the errors.

\subsection{Wrong Temperature Gradient}
While it does increase the accuracy it does also increase the needed computational effort due to the fact that to get a real added value from this,
the height out of the pressure has to be calculated at each time step with the help of the estimated pressure.
In addition when the temperature gradient is chosen wrong it deeply affects the estimation as can be seen in Figure \ref{fig:PointMassVSPressure}.
The errors there were low pass filtered with a moving average filter for better visualisation.

\begin{figure}[h!]
 \centering
 \includegraphics[width=.8 \textwidth]{./Pictures/PointMassVSPressure.jpg}
 % PointMassVSPressure.jpg: 0x0 pixel, 300dpi, 0.00x0.00 cm, bb=
 \caption{Plot of point mass vs the pressure low pass filtered}
 \label{fig:PointMassVSPressure}
\end{figure}

In this figure the error for both, a system with acceleration offset (which therefore can depend more on the accelerometer measurements)
and a system model with the pressure in the state vector are compared against each other.
While they preform equaly good as long as the temperature gradient is determined correctly,
if the gradient is determined wrong by just 5 percent it already performs less accurate than all of the other tested system models.
This is a problem due to the fact that the temperature gradient will most certainly not be correct during the whole flight.
With this it can be seen that the overall system performance of this model is more or less the same as with the normal point mass system.
On the other hand if the error on the temperature gradient is exactly known, the noise onto those measurements can be adjusted which
results in a gain of robustness for the whole estimation.

To summarize it, the pressure as a state variable is a risky but if done right a helpful adjustment of the system model.

\newpage
\section{Point Mass with Pitch angle}
The pitch angle is difficult to estimate because it has no measured dependencies on its own.
Therefore the Kalman filter does just something like a real time low pass filtering on those measurements.
This can be seen in Figure \ref{fig:PointMassPitchAnglePerformance}. The estimated pitch angle there changes slower than the generated measurements value which can be seen in chapter \ref{ch:Implementation}.

\begin{figure}[h!]
 \centering
 \includegraphics[width=.8 \textwidth]{./Pictures/PointMassPitchAnglePerformance.jpg}
 % PointMassOffsetPerformance.jpg: 0x0 pixel, 300dpi, 0.00x0.00 cm, bb=
 \caption{Performance of point mass with pitch angle over time}
 \label{fig:PointMassPitchAnglePerformance}
\end{figure}

The overall performance is more or less the same than that of the normal point mass system. This can espacially be seen in the last five seconds of the height estimation where both estimations are equal to each other (the estimation of the point mass system model does overlap the one of the system model which includes the pitch angle estimation).
The one big difference which can be seen in the plot of the first second is that the estimator tries to correct the wrong acceleration measurements (due to the offset) with the change of the pitch angle.
But because the system noise on the pitch angle states that it does not change in a great manner until the burnout the influence is restricted.

\begin{table}[h!]
\centering
\begin{tabular}{cccccc}
\hline
\multicolumn{1}{|c|}{State Variable} & \multicolumn{1}{c|}{Unit} & \multicolumn{1}{c|}{Max} & \multicolumn{1}{c|}{Min} & \multicolumn{1}{c|}{Mean} & \multicolumn{1}{c|}{Median} \\ \hline
Height                            & $m$                         & 20.24                  & 4.46e-06                 & 4.75                    & 2.24                      \\
Speed                             & $m/s$                       & 78.57                   & 0                        & 3.42                   & 2.44                      \\
Acceleration                      & $m/s^2$   			& 17.79                  & 6.26e-05                 & 1.68                    & 1.55                      \\
Angle	                  	  & $°$   			& 12.26                   & 2.46e-05                 & 2.41                  & 1.90
\end{tabular}
\caption{Error of estimated state variables point mass with pressure}
\label{tab:ErrorPointMassPitchAngle}
\end{table}

In Table \ref{tab:ErrorPointMassPitchAngle} the errors of the estimation can be seen once again.
It shows that the values are nearly the same as the ones from the point mass system.

\subsection{Small Influence}
This results in the conclusion that the influence of the pitch angle is much smaller than thought.
This mainly because the pitch angle does only start to get to a greater value after the burnout.
After this the vertical acceleration is much smaller and therefore an error of the angle by a few degrees does not have a big impact.
In other words for example if the real angle is 10 degrees, the measurement at this point is around 18 degrees while the estimation is 11 degrees.
Then this means the measured acceleration of for example 10.15$m/s^2$ should be corrected to 10$m/s^2$. With the estimated angle it is corrected to 9.97$m/s^2$,
while with the measured angle it is corrected to 9.66$m/s^2$. So for this example an error of 8 degrees would only result in a false estimation of 0.31$m/s^2$.
Although this error does not develop in a linear fashion especially when the angle has a greater value, due to the rather small accelerations after the burnout the impact of the noisy measurements is still strongly limited.
This can be seen in the plot on Figure \ref{fig:PointMassVSPitch} where the estimation which uses the pitch angle as an additional state variable makes the performance just slightly better.
\begin{figure}[h!]
 \centering
 \includegraphics[width=.8 \textwidth]{./Pictures/PointMassVSPitch.jpg}
 % PointMassVSPitch.jpg: 0x0 pixel, 300dpi, 0.00x0.00 cm, bb=
 \caption{Estimation error over time from point mass system with and without pitch angle inclusion}
 \label{fig:PointMassVSPitch}
\end{figure}

\newpage
\section{Point Mass with Acceleration as input}
The overall performance of this system model is shown in Figure \ref{fig:PointMassAccInputPerformance}.

\begin{figure}[h!]
 \centering
 \includegraphics[width=.8 \textwidth]{./Pictures/PointMassAccInputPerformance.jpg}
 % PointMassOffsetPerformance.jpg: 0x0 pixel, 300dpi, 0.00x0.00 cm, bb=
 \caption{Performance of point mass with acceleration as input over time}
 \label{fig:PointMassAccInputPerformance}
\end{figure}

Its performance does also resemble the point mass system model quite well but without any significant improvement.
As above with the pitch angle the estimated height of both systems do overlap over more or less the whole flight.
This can be seen in the upper right corner of Figure \ref{fig:PointMassAccInputPerformance} where due to the overlap only the values from the point mass system model is visible.
The difference in the errors between this system model and the normal point mass model occurs more due to rounding errors of the simulation than due to really better estimation.
This can be concluded by fact that this system model performs sometimes slightly better and sometimes slightly worse than the point mass system, therefore there is no real gain in the implementation this way.

In addition the plot of the first second of the estimation shows that the acceleration estimation has more noise on it than in the other system models.
This is because the acceleration has no model dependencies in this implementation, it is taken directly as system input.
With this the system noise on the acceleration has to resemble either the measurement or the system noise.
In this implementation it was used to bring the measurement noise into the system and therefore the low pass characteristics of the system noise is lost.
In other words with a measurement as input a tuning parameter (either system or measurement noise) which could compensate the corresponding state value cannot be used.
So there is no real gain in this implementation and it will therefore not be implemented in the final system model.

\begin{table}[h!]
\centering
\begin{tabular}{cccccc}
\hline
\multicolumn{1}{|c|}{State Variable} & \multicolumn{1}{c|}{Unit} & \multicolumn{1}{c|}{Max} & \multicolumn{1}{c|}{Min} & \multicolumn{1}{c|}{Mean} & \multicolumn{1}{c|}{Median} \\ \hline
Height                            & $m$                         & 20.05                  & 7.21e-06                 & 4.71                    & 2.18                      \\
Speed                             & $m/s$                       & 78.49                  & 0                        & 3.38                    & 2.40                      \\
Acceleration                       & $m/s^2$   			& 10.65                  & 0                        & 1.60                    & 1.43
\end{tabular}
\caption{Error of estimated state variables point mass with acceleration as input}
\label{tab:ErrorPointMassAccelerationInput}
\end{table}

\newpage
\section{Point Mass with offset and better calculated system noise}
The better calculated system noise for this estimation was calculated as stated in chapter \ref{ch:Implementation}.
This has been done by calculating the discrete system noise matrix Gd with the integration method as well as
derive the perfect measurements and then low pass filter them to get better system noise vectors.
This improvement had to be done on a system which uses the acceleration offset as well in the state vector to get the maximum possible gain out of this implementation.
This results in an overall system performance as seen in Figure \ref{fig:PointMassBetterNoisePerformance}

\begin{figure}[h!]
 \centering
 \includegraphics[width=.8 \textwidth]{./Pictures/PointMassBetterNoisePerformance.jpg}
 % PointMassOffsetPerformance.jpg: 0x0 pixel, 300dpi, 0.00x0.00 cm, bb=
 \caption{Performance of point mass with better system noise over time}
 \label{fig:PointMassBetterNoisePerformance}
\end{figure}

This system performs much like the one with acceleration offset as a state vector.
The greatest difference can be seen in the first half second where the acceleration is nearly perfectly estimated.
Also does the acceleration offset not change as much as in the system model without the improved system noise calculation.
This is because with the better system noise an optimal estimation can be achieved with less effort since the influence of the acceleration offset is more clearly defined.

\begin{table}[h!]
\centering
\begin{tabular}{cccccc}
\hline
\multicolumn{1}{|c|}{State Variable} & \multicolumn{1}{c|}{Unit} & \multicolumn{1}{c|}{Max} & \multicolumn{1}{c|}{Min} & \multicolumn{1}{c|}{Mean} & \multicolumn{1}{c|}{Median} \\ \hline
Height                            & $m$                         & 9.74	                  & 8.11e-06                 & 1.28                    & 0.73                      \\
Speed                             & $m/s$                       & 78.58                   & 0                        & 2.79                    & 1.85                      \\
Acceleration                       & $m/s^2$   			& 86.93                   & 4.07e-05                 & 3.57                    & 1.96                     \\
Acceleration Offset                & $m/s^2$   			& 88.60                   & 2.52e-05                 & 4.59                    & 2.59
\end{tabular}
\caption{Error of estimated state variables point mass with acceleration offset and better system noise}
\label{tab:ErrorPointMassBetterNoise}
\end{table}

Table \ref{tab:ErrorPointMassBetterNoise} shows that the system performs pure error wise better than each of the other system models discussed before.
While it also contains big maxima in speed, the acceleration and acceleration offset values are smaller than the ones of the system with only acceleration offset.
Due to that the mean values (except the offset) are still better than most other.

Figure \ref{fig:PointMassVSBetterNoise} shows that an overall better estimation can be achieved with this tactic.
\begin{figure}[h!]
 \centering
 \includegraphics[width=.8\textwidth]{./Pictures/PointMassVSBetterNoise.jpg}
 % PointMassVSBetterNoise.jpg: 0x0 pixel, 300dpi, 0.00x0.00 cm, bb=
 \caption{Error over time with and without better system noise}
 \label{fig:PointMassVSBetterNoise}
\end{figure}
This is mostly due to the better system noise vector which can be much better estimated this way.
Also the additional effort to access this system model only occurs on the preparation,
while it does have no effect on the computational effort during the flight itself.
So this approach of the state estimation should be used any time if available.

\newpage
\section{With and without GPS}
It is not certain that the GPS will be working in the next competitions as already stated in the introduction.
In addition the GPS sensor data needs more time to be measured and can therefore arrive too late to be included correctly.
There is the possibility of back calculation to include such too late arrived measurements into the state estimation but this needs a lot of computational effort \cite{SimonDan2006Ose:}.

Because of this, the estimation without the GPS measurements are tested with a point mass system model to find its direct impacts.
Figure \ref{fig:PointMassWithWithoutGPS} shows the plot of different estimations with and without working GPS.

\begin{figure}[h!]
 \centering
 \includegraphics[width=.8 \textwidth]{./Pictures/PointMassWihtoutGPSPerformance.jpg}
 % PointMassOffsetPerformance.jpg: 0x0 pixel, 300dpi, 0.00x0.00 cm, bb=
 \caption{Performance of point mass without GPS over time}
 \label{fig:PointMassWithoutGPSPerformance}
\end{figure}

As expected the height is further away from the ground truth in the last five seconds plot than the one estimated with GPS measurements.
Also the staircase like correction steps which come from the GPS measurements are missing.
This behaviour was expected because the the GPS measurements are most important at great heights were the remaining measurements lose their credibility.


\begin{table}[h!]
\centering
\begin{tabular}{cccccc}
\hline
\multicolumn{1}{|c|}{State Variable} & \multicolumn{1}{c|}{Unit} & \multicolumn{1}{c|}{Max} & \multicolumn{1}{c|}{Min} & \multicolumn{1}{c|}{Mean} & \multicolumn{1}{c|}{Median} \\ \hline
Height                            & $m$                         & 31.20                  & 5.77e-05                 & 8.97                    & 5.20                      \\
Speed                             & $m/s$                       & 78.58                  & 0                        & 6.35                    & 4.86                      \\
Acceleration                       & $m/s^2$   			& 17.78                  & 7.60e-05                 & 3.48                    & 2.34
\end{tabular}
\caption{Error of estimated state variables point mass without GPS measurements}
\label{tab:ErrorPointMassWithoutGPS}
\end{table}

Table \ref{tab:ErrorPointMassWithoutGPS} which displays the errors from the estimations shows that as expected the error is bigger in each value.
Also the maximum height error of over 30 meter shows that the normal state estimator is really depending on the GPS measurements.

\subsection{Wrong temperature gradient}
This system has to depend strongly on the barometer measurements to calculate the height.
Therefore the problem of a wrong temperature gradient arises once again.
To show this the estimation results with different temperature gradients are plotted in Figure \ref{fig:PointMassWithWithoutGPS}.

\begin{figure}[h!]
 \centering
 \includegraphics[width=.8\textwidth]{./Pictures/PointMassWithWithoutGPS.jpg}
 % PointMassWithWithoutGPS.jpg: 0x0 pixel, 300dpi, 0.00x0.00 cm, bb=
 \caption{Absloute error over time with and without GPS}
 \label{fig:PointMassWithWithoutGPS}
\end{figure}

It shows that especially if the temperature gradient is chosen wrong (by just 5 percent) and
therefore the height out of the pressure measurements is calculated wrong the error of the estimation rises significantly.
With this the error increases with the ascending rocket if it is not corrected by the GPS measurements.
So the aim of the best performance system has to be that it should not depend too much on those GPS measurements for a good state estimation.

\newpage
\section{Best Performing System}
Based on the results from the tests above the best performing system can be defined.
The main goal is to achieve an as robust state estimation as possible that does also match the given accuracy requirements.
The pitch angle did not have a useful positive effect on the estimation and is therefore not implemented.
Also the acceleration as input does not significantly improve the estimation.
Therefore the best system for the stated problem consist of the additional acceleration offset as a state vector as well as better calculated system noise.
Despite the fact that the pressure is risky to implement as a state variable it is used in this implementation because it does increase the robustness significantly.

\begin{figure}[h!]
 \centering
 \includegraphics[width=.8\textwidth]{./Pictures/BestSystemPerformance.jpg}
 % BestSystemPerformance.jpg: 0x0 pixel, 300dpi, 0.00x0.00 cm, bb=
 \caption{Performance over time}
 \label{fig:BestSystemPerformance}
\end{figure}

Figure \ref{fig:BestSystemPerformance} shows again the performance over one whole flight.
In the plot in the lower right corner which shows the first second of the estimation it can be seen that this system has a clearly visible settling time in the acceleration and speed.
While this seems quite big at the first sight, it is well in the given requirements of one second settling time after the burnout.
On the other hand it shows that the performance increases significantly after the acceleration and speed have settled.
This increase in the accuracy of the estimation can also be seen in the table \ref{tab:ErrorBestPerformanceSystem}.

\begin{table}[h!]
\centering
\begin{tabular}{cccccc}
\hline
\multicolumn{1}{|c|}{State Variable} & \multicolumn{1}{c|}{Unit} & \multicolumn{1}{c|}{Max} & \multicolumn{1}{c|}{Min} & \multicolumn{1}{c|}{Mean} & \multicolumn{1}{c|}{Median} \\ \hline
Height                            & $m$                         & 7.39	                  & 2.01e-06                 & 1.26                    & 0.92                      \\
Speed                             & $m/s$                       & 24.87                   & 0                        & 1.61                    & 1.26                      \\
Acceleration                       & $m/s^2$   			& 41.66                   & 4.86-06                  & 1.28                    & 0.70                     \\
Acceleration Offset                & $m/s^2$   			& 41.70                   & 5.36e-06                 & 1.01                    & 0.61                     \\
Pressure		          & $hPa$   			& 0.80                    & 2.49e-06                 & 0.14                    & 0.12
\end{tabular}
\caption{Error of estimated state variables of the best found system}
\label{tab:ErrorBestPerformanceSystem}
\end{table}

While the median of the error from the estimated height shows a slight loose in the height accuracy in comparison to the system model with better system noise alone,
the accuracy of all other values has increased.
This is especially impressive in the mean of the speed and acceleration, because they have big errors at the beginning with the shown settling problem which has been shown in the plot above.
Also the height is with a median of just 0.82 meter and a maximum of 7.39 meter still most of the time in the aimed error of 2 meter maximal error.

\subsection{Robustness}
The slight loss in accuracy was traded for an increase in robustness which is seen as more important due too different uncertainties.
They are still being able to function if a sensor fails which will be covered below as well as being robust against false system modelling
and especially falsely detected temperature gradients. The impact of those errors on the found best performing system will be shown in the following sections.

\subsubsection{Without GPS}
As stated above this system should estimate the height without a significant raise in the error without the GPS measurements.
Figure \ref{fig:ErrorWitoutGPS} shows the height error of the best system model without GPS measurements.
For comparison there is also the point mass system model estimation with and without the GPS measurements plotted.

\begin{figure}[h!]
 \centering
 \includegraphics[width=.8\textwidth]{./Pictures/ErrorPointMassBestSystemWithoutGPS.jpg}
 % ErrorPointMassBestSystemWithoutGPS.jpg: 0x0 pixel, 300dpi, 0.00x0.00 cm, bb=
 \caption{Error of point mass system and best system with and without GPS measuements}
 \label{fig:ErrorWitoutGPS}
\end{figure}

This shows that the best found system model can estimate the height as good as the point mass system without the GPS measurements.
On the other hand the point mass system model without the GPS measurements does lose accuracy as the rocket rises higher above ground.

\subsubsection{Wrong temperature gradient}
This system model should also have some robustness against a falsely defined temperature gradient.
It has to be said that in the simulation the difference of the real and the used temperature gradient is known and the noises are adjusted properly.
This will maybe not be the case in the real flight.
Table \ref{tab:ErrorChangingTempGradWithWithoutGPS} shows the errors which the state estimation makes when the temperature gradient is wrong with and without working GPS.

\begin{table}[h!]
\centering
\begin{tabular}{ccc}
\hline
\multicolumn{1}{|c|}{Temperature gradient correctness} & \multicolumn{1}{|c|}{Mean}& \multicolumn{1}{|c|}{Median} \\ \hline
Normal with GPS 	& 1.26 		& 0.92\\
Normal without GPS	& 1.43	 	& 1.11\\
5\% off with GPS 	& 2.77	 	& 2.46\\
5\% off without GPS 	& 3.39	 	& 3.24\\
10\% off with GPS 	& 6.32	 	& 7.04\\
10\% off without GPS 	& 7.74 		& 7.61
\end{tabular}
\caption{Error of the height in meter by changing temperature gradient with and without GPS measurements}
\label{tab:ErrorChangingTempGradWithWithoutGPS}
\end{table}

First of all it can be seen that with the correct temperature gradient and no GPS measurements,
the error of the height estimation is still clearly below the two meter margin.
In addition the median of the height error does only increases by around one meter if no GPS is available.
But also the robustness of this system model can be seen, if there are no GPS measurements
and the temperature gradient is 10 percent off over the whole flight it only results in a median error of the height estimation of 7.61 meter.

\section{Sensor Outfall}
In addition to the tests for the different system model implementation
the possibility to simulate outfalls of the different sensors at start or during the flight was implemented for the last system model.
For this the corresponding measurement noise of this sensors values can be set to infinity at given time step.
The most possible way was found that a sensor fails during the burning of the motor due to the vibration.
Therefore the here discussed scenarios are all the same with the corresponding sensor failing at the third second of the flight while the motor is burning.

\subsection{GPS Outfall}
The outfall of the GPS was already discussed above which would represent a state estimation with no GPS measurements at all.
If the GPS falls out during the flight the behaviour of the state estimation should resemble the one above after the outfall.
The test have shown that this is the case.
If the GPS falls out at second three the mean of the height error rises to 1.42 meter while the median rises to 1.11 meter.
The other state variable errors rise too by a small amount but as expected the pressure estimation error stays the same.
These values show that the estimation is just slightly better than without any GPS at all.

\subsection{Barometer Outfall}
The barometer is special because if all barometers are lost,
the height which is calculated out of the state vector pressure has also to be set to failed.

\subsubsection{One Barometer fails}
The loss of one barometer does already make a significant change into the estimation.
For example the mean of the height error rises to 1.78 meter while the median also rises to 1.39 meter.
The difference between the mean and the median shows that there occur more outliers if just one barometer is active.
This can also be seen in the error of the barometer estimation which rises to 0.2 hP for the mean and 0.16 hP for the median.

\subsubsection{All Barometer fail}
More interesting is how the state estimator is performing when no barometer measurements are available at all.
As expected the error of the barometer estimation does rise onto great values which are around 12 hP for the mean and 11 hP for the median.
But on the other hand the accuracy of the height estimation does actually rise to more or less the same values as with two working barometers.
This behaviour can be explained with the still working GPS.
If both barometers fall out and therefore there is no height calculated out of the pressure, the height from the
GPS sensor gets a higher weight due to it being the only remaining measurements on the height.

\newpage
\subsection{Accelerometer Outfall}
The influence of an outfall of the accelerometer can be well visited in the plots of Figure \ref{fig:PerformanceAccOutfall}.

\begin{figure}[h!]
 \centering
 \includegraphics[width=.8\textwidth]{./Pictures/BestSystemPerformanceAccOutfall.jpg}
 % ErrorPointMassBestSystemWithoutGPS.jpg: 0x0 pixel, 300dpi, 0.00x0.00 cm, bb=
 \caption{Performance over a whole flight with failing accelerometer at second 3}
 \label{fig:PerformanceAccOutfall}
\end{figure}

It shows how the acceleration estimation starts to change in great manner after the third second.
But as it can be seen in the plot of the last five seconds the height estimation does not change in a great manner
compared to the estimation with acceleration measurements during the whole flight.
The calculated height errors do confirm this by values of 1.54 meter for the mean and 1.20 meter for the median.
The pressure estimation is not impacted as it was expected.
Also the error of the acceleration does rise by around 10 $m/s^2$ and does therefore also effect the speed estimation.

\subsection{Gyrometer Outfall}
Due to the fact that the pitch angle does not have such a great impact onto the acceleration as already stated above,
the effect onto the state estimation if the gyrometer fails is also quite small.
First of all the error which is made in the estimation of the acceleration should be examined.
Against the expectation it does rise by around 1$m/s^1$ for each the mean and also the median value.
Due to that the estimation of the height does also lose some accuracy and is with 1.39 meter for the mean
and 1.11 meter for the mean resemble the same situation as if there would be no GPS measurements.
This shows that while it is not necessary to estimate the pitch angle the measurements of the gyrometer should still be included for an optimal estimation.

\newpage
\subsection{Multiple Sensor Outfall}
A requirement which was stated is that the sensor fusion should be able to work without 2-3 sensors.
Two sensors was already discussed like if both barometer or the accelerometer
(which would also resemble the outfall of the gyrometer, since those measurements would not be included either) fails.
An additional test should therefore be what happens when three sensor fail.
For this scenario the two barometers and the GPS were chosen to fail because if they fail the observably of the point mass system is still secured.

\begin{figure}[h!]
 \centering
 \includegraphics[width=.8\textwidth]{./Pictures/BestSystemPerformanceAccOutfall.jpg}
 % ErrorPointMassBestSystemWithoutGPS.jpg: 0x0 pixel, 300dpi, 0.00x0.00 cm, bb=
 \caption{Performance over a whole flight with both barometer and the gyrometer failed}
 \label{fig:PerformanceMOutfall}
\end{figure}

With those failed sensors the estimation does still work quite well which can be seen in the plots of the figure \ref{fig:PerformanceMOutfall}.
This is because the GPS which provides the most accurate measurements is still working.
Due to that the error of the height estimation does just rise slight amount to 1.54 meter for the mean and 0.97 meter for the median.
For comparison if the GPS sensor does also fail the height estimation does lose accuracy by a great amount and so the error rises to 63 meter for the mean and 26 meter for the median.
Also it should be said that the height error does end to 531.78 meter which is quite far off.


\chapter{Conclusion}
\label{ch:Conclusion}
\section{Derived Solution}
Here comes the best finded state estimator loop/solution and how an why it works
Again Just a quick reminder due the main solution derivision was made in the tests.


\section{Comparison Solution/Requirements}
Compare the derived solution to the requirements

\section{Outlook}
Different things which can be done differently in the future
- Exctented Kalman filter
- More better documented test flights.
- Back calculation for later mesurements espacially GPS measurements
- Slower sampling

\section{Thanks}
Thanks to the Aris Team espacially Thomas and Fabian
Also Thanks to Lukas and Papa Moll

And to the EPFL team for the test flight data



\bibliography{References}



\begin{appendices}
\chapter{Assignment}
\includepdf[pages=-]{Pictures/Aufgabenstellung.pdf}

\chapter{Lists}
\listoffigures

\listoftables




\end{appendices}



\end{document}
