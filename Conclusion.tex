\section{Derived Solution}
The solution for the stated problem which was found over this thesis can be summaryzed as follows.
The algorithm uses a discrete Kalman-filter which has dynamic noise matrices for the measurements as well as the system.
In the measurements noise matrices are the noises for the GPS and the Pressure set to infinity while no measurements from the GPS module or the Barometers are available.
The same concept is used if it is detected that a sensor is not working properly.
The state vector which describes the estimated system consist of the height, speed, acceleration, acceleration offset and the pressure.
To work in the optimal possible test flights with the used sensors have to be made and the noise matrices have to be calculated out of the measured data.
This provides an optimal trade off between robustness and accuracy and a small computational effort as possible.

\section{Comparison Solution/Requirements}
To how good the given requirements are met can be described by filling out the requirements table from the chapter \ref{ch:Introduction}



\section{Outlook}
As always there are points which could be developed further for better results which are stated here.
\subsection{Back calculation for later mesurements espacially GPS measurements}
\subsection{}
Different things which can be done differently in the future
- Exctented Kalman filter
- More better documented test flights.
- Back calculation for later mesurements espacially GPS measurements
- Slower sampling
- Calculating the pressure noise as good as possible find out how to detect how rwong the temperatur gradient is

\section{Thanks}
First of all credits have to be given to the whole ARIS team which supported this thesis with support
and great parts of a already functioning simulation which could be used to generated the trajectories.
Special thanks goes to Thomas bla and Fabian bla which provided there knowledge and the logging data from the test flights.
But also the EPFL team of the Matterhorn competition has to be thanked for the provided logging data.

\section{Reflection}
This thesis consist of theory like no one before that I have written.
Due to this, it was a quite new experience. 
This because the whole thesis consisted off concepts and there implementation in a simulation.
