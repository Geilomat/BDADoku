 \documentclass[main.tex]{subfiles}


 \cite{SimonDan2006Ose:}

 
 \section{Bla}
 About Sensor Fusion and ARIS
 
 
 \section{Research}
 Write about the Papers/book you used:
 Kalman-filter
 Optimal state estimation
 The Master Thesis
 
 \section{Problem}
 State the problem, what will be difficult ? For what is it ? where should it be improved
 
 Problems found so far:
 \begin{itemize}
  \item How to calculate Height out of Pressure/Temp/Humidity Fabian ver: $44330 * (1 - (\frac{pressure}{101325})^{ \frac{1}{5.255}})$
  \item How to integrate AirBreaks/Drag Force of Air/ Trust of Motor ?
  \item The up-flight is rather short: about 25s so the Fusion should have a small settling time
  \item The Chip on which it is used is no the fastes : 168 MHz clock
  \item The Ram on the Chip is not endless: Maximal space for the Sensor fusion is about 10kB
  \item The Sensor Fusion should be as modular as possible so that it also can be used in the next competition
  \item The Sensor Fusion has to be as sturdy as possible so that it will not fail if a problem occurs
  \item The Fusion should make a state Estimation as percicse as possible.
  \item There are a lot of different variables: 3 Positions, 1 Speed, 3 Accelorations, 3 Lagen, Time, Pressure, Tempterature, Humidity, Up-/Downforce.
  \item Especially the Input Value u which is the force onto the rocket is difficult to define (Drag, Trust = acceloration depends on wheigt which changes over time).
  \item The different Sensor have different weaknesses: \begin{itemize}
							 \item Accelometer: Offset, drift, weak to vibrations
                                                         \item Gyro: Weak to Vibrations
                                                         \item Barometer: Many uncertenties, unpercise
                                                         \item GPS: Slow (max 5Hz)
                                                        \end{itemize}
                                                        
								  
 \end{itemize}

 
 
\section{Requirements}
These are the requirements which were drown out of the problems list.
 
 \begin{table}[h]
\centering
\begin{tabular}{|l|l|l|l|}
\hline
Requirement   & \multicolumn{1}{c|}{Rating} & Aim & Importance \\ \hline
Precision     & Error between estimation and groundtruth  & < 5\% after settling time & High  \\ \hline
Sturdiness    & \% Percision without \# sensors  & -10\% per failed sensor  & Medium \\ \hline
Code Size     & kB of RAM needed  & < 10kB size & High \\ \hline
System Load   & Time per loop on the cuurent Chip (128 MHz) & < 1ms & Crictal \\ \hline
Non Linearity & \# non linearities in the algortihm  & 0 &  Desirable\\ \hline
Settling time & Time from ignition to optimal estimation  & < 5s       &  Critical\\ \hline
Modularity    & Effort needed to integrate a new sensor    & < 10h work     &  Desirable\\ \hline
              &                            &            &  \\ \hline
              &                            &            &  \\ \hline
              &                            &            &  \\ \hline
\end{tabular}
\caption{Requirements table}
\label{tab:Requirements}
\end{table}
 
 \section{Desired Solution}
 
 Describe in quick terms what you are aiming for.