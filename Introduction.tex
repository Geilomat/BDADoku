

 Here comes the Introduction what is Aris what are the aims and why is a Data/Sensor fusion needed? 
 
 The Academic Space Initiative Switzerland (ARIS) is student group which tries too compete in the yearly Intercollegiate Rocket Engineering Competition (IREC).
 To aim for the right apogee (3000 m) a Control algorithm is implemented. This algorithm relays on the information of different sensors to determine the rockets actual state.
 Cause there are different sensors to measure the same value a algorithm which fuses those data would come in handy.
 With this fusion algorithm it should also be possible to be more accurate as with each sensor on its own.
 For this the problem as well as the desired solution will be defined in this chapter.
 After that the dynamics of the rocket as well as the parameter of the different Sensors are defined at beginning of chapter \ref{ch:Approach}, the most suited algorithm will be chosen.
 Chapter \ref{ch:Implementation} will then describe how this fusion algorithm implemented in a simulation in detail.
 To verify that the implementation is working as intended, the fused data will be held against the ground truth which are provided from this years test flights in chapter \ref{ch:Tests}.
 In the last chapter \ref{ch:Conclusion} a summary of the achieved knowledge will be stated.
 The purpose of this thesis is to find and implement the algorithm which is most suitable for this task.
 
 \section{Purpose}
 The hardware as well as the most of the software parts that will be used for this competition is already defined.
 Also it is a suitable assumption that the sensors and the dynamics of the rocket will be stay more or less the same for the competitions coming.
 Therefore this thesis will mainly focus on finding a algorithm for this given surroundings, but it is also will try to find as modular solution as possible, so that achieved knowledge can be used in further competition.
  
 \section{Research}
 Write about the Papers/book you used:
 Kalman-filter
 Optimal state estimation
 The Master Thesis
 Sensor/data fusion is an engineering field since the first rocket flights. Therefore there is already a lot of previous work which can be used in this thesis.
 
 
 \section{Problem}
 State the problem, what will be difficult ? For what is it ? where should it be improved
 
 Problems found so far:
 \begin{itemize}
  \item How to calculate Height out of Pressure/Temp/Humidity Fabian version: $44330 * (1 - (\frac{pressure}{101325})^{ \frac{1}{5.255}})$
  \item How to parameterize the different sensors (Measuring, Test Flight, Data Sheet ) ? 
  \item How to fuse together Data from Sensors that have different Taus, especially those who are slower than the Loop-Time ?
  \item How to integrate AirBreaks/Drag Force of Air/ Trust of Motor a input value?
  \item What are the different noise factors and when do they occur ?
  \item The up-flight is rather short: about 25 seconds, so the Fusion should have a small settling time
  \item The Micro-Chip on which it is used is no the fastest : 168 MHz clock
  \item The Ram on the Chip is not endless: Maximal space for the Sensor fusion is about 10kB
  \item The Sensor Fusion should be as modular as possible so that it also can be used in the next competition
  \item The Sensor Fusion has to be as sturdy as possible so that it will not fail if a problem occurs
  \item The Fusion should make a state Estimation as precise as possible.
  \item There are a lot of different variables: 3 Positions, 1 Speed, 3 Accelerations, 3 Lagen, Time, Pressure, Tempterature, Humidity, Up-/Downforce.
  \item Especially the Input Value u which is the force onto the rocket is difficult to define (Drag, Trust = acceloration depends on wheigt which changes over time).
  \item The different Sensor have different weaknesses: \begin{itemize}
							 \item Accelerometer: Offset, drift, weak to vibrations
                                                         \item Gyro: Weak to Vibrations
                                                         \item Barometer: Many uncertenties, unpercise
                                                         \item GPS: Slow (max 5Hz)
                                                        \end{itemize}
                                                        
								  
 \end{itemize}

 \cite{}
 
\section{Requirements}
These are the requirements which were drown out of the problems list.
 
 \begin{table}[h]
\centering
\begin{tabular}{|l|l|l|l|}
\hline
\bf{Requirement}   & \bf{Rating} & \bf{Aim} & \bf{Importance} \\ \hline
Precision     & Error between estimation and ground truth  & < 2m in Z & High  \\ \hline
Reliability   & Functioning Estimation with \#failed sensors & Functions without 2-3 sensors & Medium \\ \hline
System Load   & \# Calculation steps per loop & < 1000 & Critical \\ \hline
Non Linearity & \# non linearity's in the algorithm  & 0 &  Desirable\\ \hline
Settling time & Time from ignition to optimal estimation  & < 5 s &  Critical\\ \hline
Modularity    & Effort needed to change a sensors & < 10 h work &  Desirable\\ \hline

\end{tabular}
\caption{Requirements table}
\label{tab:Requirements}
\end{table}
 
\section{Desired Solution}
 
Describe in quick terms what you are aiming for.